\section{Gestion des FAQ}

\begin{center}
\scalebox{0.7}{\includegraphics{images/FAQ.jpg}}\\
\par{Package Gestion Faq}
\end{center}
Voici les diff{\'e}rents sc{\'e}narios:\\
	\begin{itemize}
	\item Consultant :
		\begin{itemize}
			\item Consulter une FAQ d'un exercice:
			\begin{itemize}
			\item Pr{\'e}-requis : Etre log{\'e}/identifi{\'e}.\\
			L'exercice poss{\`e}de une FAQ.
			\item Description :  L'utilisateur s{\'e}lectionne un exercice.\\
			L'utilisateur consulte le fichier FAQ associ{\'e} {\`a} l'exercice.
			\item Post-requis : La FAQ de l'exercice s{\'e}lectionn{\'e} est affich{\'e}e.
			\end{itemize}		
		\end{itemize}

	\item Enseignant
		\begin{itemize}
		\item Associer une FAQ {\`a} un exercice:
			\begin{itemize}
			\item Pr{\'e}-requis : Etre log{\'e}/identifi{\'e}.\\
			L'exercice ne poss{\`e}de pas encore de FAQ.
			\item Description :  L'utilisateur selectionne un exercice.\\
			L'utilisateur clique l'option {\it Ajouter une FAQ}.\\
			L'utilisateur choisit un fichier qui se trouve sur son compte.
			L'utilisateur importe un fichier FAQ et l'associe {\`a} cet exercice.
			\item Post-requis : L'exercice poss{\`e}de une FAQ associ{\'e}e.
			\end{itemize}

		\item D{\'e}sassocier une FAQ d'un exercice:
			\begin{itemize}
			\item Pr{\'e}-requis : Etre log{\'e}/identifi{\'e}.\\
			L'exercice poss{\`e}de une FAQ.
			\item Description :  L'utilisateur selectionne un exercice.\\
			L'utilisateur clique sur l'option {\it Supprimer une FAQ}.
			\item Post-requis : L'exercice ne poss{\`e}de plus de FAQ associ{\'e}e.\\
			\end{itemize}
		\end{itemize}
	\end{itemize}

Voyons maintenant les priorit{\'e}s des diff{\'e}rentes actions quant au
d{\'e}veloppement du logiciel.\\\\\\
\begin{tabular}{|p{4cm}|c|p{4cm}|p{5cm}|}
\hline
Fonction & Priorit{\'e} & Qualit{\'e} & Mesure \\
\hline
Consulter la FAQ d'un exercice & 1 & Lisibilit{\'e} & Bonne distinction entre les questions et les r{\'e}ponses gr{\^a}ce {\`a} l'utilisation de couleurs.\\
\hline
Associer une FAQ {\`a} un exercice & 1 & Facilit{\'e} & Utilisation de menus simples\\
\hline
D{\'e}sassocier une FAQ {\`a} un exercice & 1 & Facilit{\'e} & Action simple sans archivage\\
\hline
\end{tabular}\\

\begin{center}
{\'e}chelle de mesure de la priorit{\'e}:

\scalebox{0.5}{\includegraphics{images/echelle.jpg}}
\end{center}