\section{Identification}

\begin{center}
\scalebox{0.7}{\includegraphics{images/identification.jpg}}\\
\par{Paquetage Gestion des Comptes}
\end{center}
Voici les diff{\'e}rents sc{\'e}narios:\\

\section*{Enseignant}

\begin{tabular}{|p{4cm}|c|p{4cm}|p{5cm}|}
\hline
  Fonction & Priorit{\'e} & Qualit{\'e} & Mesure \\
\hline
Consulter son compte & 3 & S{\'e}curit{\'e} & \\
\hline
Se loger & 5 & S{\'e}curit{\'e}. Fiabilit{\'e} & Se connecter au compte entr{\'e}.\\
\hline
Se d{\'e}loger  & 5 & S{\'e}curit{\'e}. Fiabilit{\'e} &  D{\'e}connexion assur{\'e}e. Ne pas pouvoir d{\'e}loger en m{\^e}me temps d'autres personnes connect{\'e}es sur l'application.\\
\hline
Modifier param{\`e}tres compte  & 3 & s{\'e}curit{\'e},fiabilit{\'e} & Garantir une 
  s{\'e}curit{\'e}. Le nouveau mot de passe ne peut pas {\^e}tre intercept{\'e}. Garantir la validit{\'e} du mot de passe  \\
\hline
\end{tabular}
\begin{center}
{\'e}chelle de mesure de la priorit{\'e}:

\scalebox{0.5}{\includegraphics{images/echelle.jpg}}
\end{center}

\begin{itemize}
\item  {\bf Se loger :}
	\begin{itemize}
	\item Pr{\'e}-requis : Avoir un compte et un mot de passe
	\item Description : L'utilisateur remplit les
	champs Login et Mot de passe dans la page d'accueil.\\
	Il valide.
	\item Post-requis : L'utilisateur est identifi{\'e} dans l'intranet.
	\end{itemize}
\item  {\bf Se d{\'e}loger :}
	\begin{itemize}
	\item Pr{\'e}-requis : Etre identifi{\'e}/log{\'e}. 
	\item Description : L'utilisateur clique sur
	{\it Se d{\'e}connecter} dans la page des liens.
	\item Post-requis : L'utilisateur se trouve dans la page d'accueil de l'intranet.
	\end{itemize}
\item  {\bf Consulter son compte :}
	\begin{itemize}
	\item  Pr{\'e}-requis : Etre identifi{\'e}.
	\item Description : L'utilisateur clique sur
	{\it Consulter mon compte} dans la page des liens. 
	\item Post-requis : Affichage de la page de
	consultation du compte de l'utilisateur connect{\'e}.
	\end{itemize}
\item  {\bf Modifier les param{\`e}tres :}
	\begin{itemize}
	\item Pr{\'e}-requis : Etre identifi{\'e}/log{\'e}. 
	\item Description : L'utilisateur clique sur
	{\it Modifier} lors de la consultation de son compte.\\
	Dans le cas du mot de passe : il entre son
	nouveau mot de passe et le confirme.
	\item Post-requis : Les param{\`e}tres sont modifi{\'e}s.
	\end{itemize}
\end{itemize}

\section*{Administrateur}

\begin{tabular}{|p{4cm}|c|p{4cm}|p{5cm}|}
\hline
  Fonction & Priorit{\'e} & Qualit{\'e} & Mesure \\
\hline
Ajouter un compte &  5 & Fiabilit{\'e} & Coh{\'e}rence entre les logins et les mots de passe. \\
\hline
Modifier un compte & 3 & Fiabilit{\'e} & Coh{\'e}rence entre les logins et les
  mots de passe. \\
\hline
Rechercher un compte & 5 & Fiabilit{\'e} & Coh{\'e}rence entre la requ{\^e}te demand{\'e}e et la r{\'e}ponse donn{\'e}e. \\
\hline
Consulter un compte quelconque & 2 & Sur. S{\'e}curit{\'e} & Eviter toutes erreurs de l'administrateur. \\
\hline
Supprimer un compte & 4 & S{\'e}curit{\'e}. Fiabilit{\'e} & Ne pas supprimer
  d'autres comptes que celui s{\'e}lectionn{\'e}.\\
\hline
\end{tabular}

\begin{itemize}
\item  {\bf Ajouter compte :}
	\begin{itemize}
	\item Pr{\'e}-requis : Etre identifi{\'e}/log{\'e}. 
	\item Description : L'administrateur clique sur le lien {\it Ajout de compte}.\\
	Il entre dans une nouvelle page o{\`u} il doit
	remplir des champs pour le nouveau compte.\\
	Il entre un login et ajoute un mot de passe qu'il confirme.\\
	Il a la possibilit{\'e} de g{\'e}n{\'e}rer un login et
	mot de passe automatiquement (le login est g{\'e}n{\'e}r{\'e} {\`a} partir du
	nom et pr{\'e}nom de l'utilisateur. Le mot de passe est g{\'e}n{\'e}r{\'e}
	al{\'e}atoirement).\\
	Il entre {\'e}ventuellement les param{\`e}tres
	personnelles de l'utilisateur.
	\item Post-requis : Nouveau compte ajout{\'e} dans
	la base de donn{\'e}es.
	\item {\bf Remarque :} L'enseignant poss{\'e}dant
	le compte ainsi cr{\'e}{\'e} n'est rattach{\'e} {\`a} aucun
	enseignement, il peut ainsi juste
	cr{\'e}er/modifier/supprimer des exercices (s'il
	est propri{\'e}taire pour la suppression et la modfication).  
	\end{itemize}

\item  {\bf Rechercher compte :}
	\begin{itemize}
	\item Pr{\'e}-requis : Etre identifi{\'e}/log{\'e}. 
	\item Description : L'administrateur clique
	sur le lien {\it Rechercher compte}. Il a la
	possibilit{\'e} de rechercher soit par nom soit
	par login ou peut tout simplement lister tous
	les comptes existants (affichage par ordre
	alphab{\'e}tique des noms ou login, au choix).
	\item Post-requis : Un listing du ou des
	comptes recherch{\'e}s s'affiche(nt) s'il(s)
	existe(nt) dans la base.
	\end{itemize}

\item  {\bf Modifier compte :}
	\begin{itemize}
	\item Pr{\'e}-requis : Etre log{\'e}/identifi{\'e}. 
	\item Description : L'administrateur recherche
	le compte voulu puis clique sur le lien {\it
	Modifier} ({\`a} c{\^o}t{\'e} du lien vers le compte).\\
	(Il a la possibilit{\'e} de modifier le compte lors
	de la consultation.)
	Il entre dans la page de modification de compte.\\
	Il a la possibilit{\'e} de modifier les param{\`e}tres
	du compte.\\
	L'administrateur a la possibilit{\'e}
	d'ajouter/retirer des droits sur des
	enseignements (voir dans {\bf Gestion des droits}).
	\item Post-requis : Le compte est modifi{\'e} dans la base. 
	\end{itemize}

\item  {\bf Consulter un compte quelconque:}
	\begin{itemize}
	\item Pr{\'e}-requis : Etre log{\'e}/identifi{\'e}. 
	\item Description : L'administrateur recherche
	le compte voulu puis clique sur le lien {\it
	Consulter} {\`a} c{\^o}t{\'e} du compte.\\
	Il entre dans la page de consultation du compte.
	\item Post-requis : Il obtient la page de consultation.\\
	La base de donn{\'e}e reste inchang{\'e}.
	\end{itemize}

\item  {\bf Supprimer compte :}
	\begin{itemize}
	\item Pr{\'e}-requis : Etre log{\'e}/identifi{\'e}.
	\item Description : L'administrateur recherche
	le compte voulu puis clique sur le lien {\it
	Supprimer} {\`a} c{\^o}t{\'e} du compte.\\
	Il est redirig{\'e} vers une page de
	confirmation o{\`u} il valide.\\ Il a la
	possiblit{\'e} de supprimer le compte lors de la consultation.
	\item Post-requis : Suppression du compte
	selectionn{\'e} dans la base de donn{\'e}e\\
	\end{itemize}
\end{itemize}
