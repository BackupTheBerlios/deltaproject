\chapter{Annexe}

\section*{ Glossaire}
\par 
\par \begin{center} {\huge \bf A} \end{center}
\par{\bf Administrateur} : Personne qui a les super pouvoirs, g�re plus
pr�cis�ment les comptes utilisateurs.\\
\\
\par \begin{center} {\huge \bf B} \end{center}
\par{\bf Base de donn�es} : Syst�me de stockage d'information.\\
\\
\par \begin{center} {\huge \bf C} \end{center}
\par{\bf Compte} : Association d'un login avec un mot de passe et de
droits sp�cifiques.\\
\par{\bf Consultant}: Personne qui visualise ou t�l�charge les TDs ou projet.\\
\par \begin{center} {\huge \bf D} \end{center}
\par{\bf Se d�loger } : Retourner � la page d'accueil en tant que
simple consultant .\\
\\
\par \begin{center} {\huge \bf E} \end{center}
\par{\bf Enseignant}: Personne qui g�re les exercices et les projets.\\
\par{\bf Enseignement}: repr�sente une mati�re, une ann�e et une section.\\
\par{\bf Exercice }: constitue un ensemble d'enonc� et de questions.\\
\\
\par \begin{center} {\huge \bf F} \end{center}
\par{\bf FAQ }(foire aux questions) : Fichier contenant les questions les plus fr�quement
pos�es et leurs r�ponses ( d�finies par l'enseignant).\\
\\
\par \begin{center} {\huge \bf I} \end{center}
\par{\bf Intranet } : R�seau accessible seulement � l'interieur de
l'Universit�.\\
\\
\par \begin{center} {\huge \bf L} \end{center}
\par{\bf Se loger } : S'identifier par rapport au serveur.\\
\\

\par \begin{center} {\huge \bf P} \end{center}
\par{\bf Param�tres de compte}:le param�tre de compte c'est le login, les
droits et autres informations n�cessaires � la personnalisation d'un
compte.\\
\par{\bf Projet}: constitue un seul exercice (g�n�ralement assez gros).\\
\\
\par \begin{center} {\huge \bf R} \end{center}
\par{\bf Responsable}: Personne pouvant rajouter et supprimer des enseignements.\\
\\
\par \begin{center} {\huge \bf T} \end{center}
\par{\bf TD}: constitue un ensemble d'exercices.\\





  