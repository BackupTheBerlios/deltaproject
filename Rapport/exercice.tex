\section{Gestion des exercices}

Voici les diff{\'e}rents sc{\'e}narios:

\section*{Enseignant}

\begin{center}
\scalebox{0.6}{\includegraphics{images/Exercice_Enseignant.jpg}}\\
\end{center}

\begin{tabular}{|p{4cm}|c|p{4cm}|p{5cm}|}
\hline
  Fonction & Priorit{\'e} & Qualit{\'e} & Mesure \\
\hline
Cr{\'e}er un exercice & 1 & Facile et Fiable & Pas de balises {\`a}
  sp{\'e}cifier. Il faut que l'exercice soit tel que le d{\'e}sire
  l'enseignant.\\
\hline
Importer un exercice & 5 & Fiable, rapide & Ne pas g{\'e}n{\'e}rer d'erreur lors du transfert du fichier.\\
\hline
Modifier un exercice & 3 & Suret{\'e} & L'ancien exercice doit {\^e}tre correctement archiv{\'e} pour permettre de revenir en arri{\`e}re.\\
\hline
D�placer un exercice vers la corbeille & 4 & Fiable & L'archivage doit {\^e}tre effectu{\'e}.\\
\hline
Rechercher un exercice & 3 & Rapide & M{\'e}thode d'acc{\`e}s rapide avec mots
  cl{\'e}s.\\
\hline
Consulter un exercice & 4 & Rapidit{\'e} et Lisibilit{\'e} & Chargement
  rapide. Affichage clair.\\
\hline
\end{tabular}

\begin{center}
{\'e}chelle de mesure de la priorit{\'e}:
\scalebox{0.5}{\includegraphics{images/echelle.jpg}}
\end{center}
\newpage
\begin{itemize}
\item {\bf Cr{\'e}er un exercice :}
	\begin{itemize}
	\item Pr{\'e}-requis : Etre log{\'e}/identifi{\'e}.
	\item Description : L'utilisateur choisit un enseignement.\\
	Il utilise l'{\'e}diteur d'exercice de l'intranet pour cr{\'e}er un exercice.\\
	Il choisit les mots clefs pour l'indexation de cet exercice.
	\item Post-requis : La base des exercices contient un exercice de plus.\\
	\end{itemize}

\item  {\bf Importer exercice : }
	\begin{itemize}
	\item Pr{\'e}-requis : Etre log{\'e}/identifi{\'e}.
	\item Description : L'utilisateur choisit un enseignement.\\
	L'utilisateur poss{\`e}de un exercice en XML.
	L'utilisateur importe cet exercice.\\
	Il choisit les mots clefs pour l'indexation de cet exercice.
	\item Post-requis : La base des exercices contient un exercice de plus.\\
	\end{itemize}

\item  {\bf Modifier exercice :}
 	\begin{itemize}
	\item Pr{\'e}-requis : Etre log{\'e}/Identifi{\'e}.\\
	L'exercice que l'utilisateur veut modifier existe.
	\item Description : L'utilisateur choisit un exercice.\\
	Il clique l'option {\it Modification}\\
	Il choisit le nouvel exercice 
	\item Post-requis : L'ancien exercice est archiv{\'e}.\\
	\end{itemize}

\item  {\bf Rechercher exercice :}
	\begin{itemize}
	\item Pr{\'e}-requis : Etre log{\'e}
	\item Description : L'utilisateur rentre les mots clefs dans un moteur de recherche puis valide son choix.\\
	Une fen{\^e}tre affiche tous les exercices qui contiennent les mots clefs ins{\'e}r{\'e}s.\\
	L'utilisateur peut cliquer sur un lien contenu dans la fen{\^e}tre pour consulter l'exercice de son choix.
	\item Post-requis : L'utilisateur peut choisir parmis les r{\'e}ponses de la recherche.\\
	\end{itemize}

\item  {\bf Consulter un exercice :}
	\begin{itemize}
	\item Pr{\'e}-requis : Etre log{\'e}
	\item Description : L'utilisateur effectue un listing de tous les exercices ou une recherche.\\
	L'utilisateur choisit l'exercice de son choix.\\
	Il peut consulter l'exercice et les corrections existantes lui correspondants.
	\item Post-requis : L'utilisateur visualise l'exercice choisi.\\
	\end{itemize}
\end{itemize}

\section*{Administrateur}

\begin{center}
\scalebox{0.7}{\includegraphics{images/Exercice_Administrateur.jpg}}\\
\end{center}

\begin{tabular}{|p{4cm}|c|p{4cm}|p{5cm}|}
\hline
  Fonction & Priorit{\'e} & Qualit{\'e} & Mesure \\
\hline
Supprimer un exercice & 4 & Fiable & L'archivage doit {\^e}tre effectu{\'e} et
  ne pas supprimer l'exercice s'il apparait dans un autre devoir.\\
\hline
Restaurer un exercice & 3 & Fiable & L'exercice doit �tre tel qu'il l'�tait avant son passage dans la corbeille.\\
\hline
\end{tabular}
\begin{center}
{\'e}chelle de mesure de la priorit{\'e}:
\scalebox{0.5}{\includegraphics{images/echelle.jpg}}
\end{center}

\begin{itemize}
\item  {\bf Supprimer un exercice :}
	\begin{itemize}
	\item Pr{\'e}-requis : Etre log{\'e}/identifi{\'e}.\\
	L'exercice que l'utilisateur veut supprimer doit se trouver dans la corbeille.
	\item Description : L'utilisateur choisit un exercice.\\
	Il clique l'option {\it Suppression}\\
	Une fen{\^e}tre de confirmation apparait. L'utilisateur confirme son choix.
	\item Post-requis : L'exercice est supprim{\'e} de la corbeille.\\
	\end{itemize}
\item  {\bf Restaurer un  exercice :}
	\begin{itemize}
	\item Pr{\'e}-requis : Etre log{\'e}/identifi{\'e}.\\
	L'exercice que l'utilisateur veut retaurer doit se trouver dans la corbeille.
	\item Description : L'utilisateur choisit un exercice.\\
	Il clique l'option {\it Restaurer}. Une fen{\^e}tre de confirmation apparait et l'utilisateur confirme son choix.\\
	L'exercice reprend la place exact qu'il avait avant d'�tre d�plac� dans la corbeille.
	\item Post-requis : L'exercice est restaur�.\\
	\end{itemize}

\end{itemize}