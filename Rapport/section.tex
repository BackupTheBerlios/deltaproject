\section{Gestion des sections}

\begin{center}
\scalebox{0.7}{\includegraphics{images/section.jpg}}\\
\par{Package Gestion Section}
\end{center}
Voici les diff{\'e}rents sc{\'e}narios:\\
	\begin{itemize}
	\item EnseignantAdmin
		\begin{itemize}
		\item Cr{\'e}er Section :
			\begin{itemize}
			\item Pr{\'e}-requis : Etre log{\'e} 
			\item Description :Il s'identifie avec son login et son mot de passe. \\
			Il saisie la nouvelle section et valide.
			\item Post-requis : Une nouvelle section est cr{\'e}{\'e}e si elle n'existait pas.
			\end{itemize}

		\item Modifier section :
			\begin{itemize}
			\item Pr{\'e}-requis : Etre log{\'e} et cr{\'e}ateur de la section.
			\item Description : Il s'identifie avec son login et son mot de passe.\\
			Il selectionne la section de son choix et clique l'option {\it Modification}.\\
			Il effectue les diff{\'e}rentes modifications qu'il souhaite.\\
			Il valide ses modifications.\\
			Seul le cr{\'e}ateur de la section peut la modifier.
			\item Post-requis : La section est modifi{\'e}e.
			\end{itemize}

		\item Supprimer section :
			\begin{itemize}
			\item Pr{\'e}-requis : Etre log{\'e} et cr{\'e}ateur de la section.
			\item Description : Il s'identifie avec son login et son mot de passe.
			Il selectionne la section de son choix et clique l'option {\it Suppression} et valide.\\
			Une fen{\^e}tre de confirmation apparait, il confirme son choix
			\item Post-requis : On archive tous les enseignements qui faisait partie de cette section.
			\end{itemize}
		\item Lister les sections :
			\begin{itemize}
			\item Pr{\'e}-requis : Etre log{\'e}.
			\item Description : Il s'identifie avec son login et son mot de passe.\\
			Il demande le listing des sections.\\
			Elles apparaissent dans une fen{\^e}tre.
			\item Post-requis : Il peut effectuer une action {\it Modification, Suppression} sur la section s{\'e}lectionn{\'e}e.\\
			\end{itemize}
		\end{itemize}
	\end{itemize}

Voyons maintenant les priorit{\'e}s des diff{\'e}rentes actions quant au
d{\'e}veloppement du logiciel.\\\\\\
\begin{tabular}{|p{4cm}|c|p{4cm}|p{5cm}|}
\hline
  Fonction & Priorit{\'e} & Qualit{\'e} & Mesure \\
\hline
Ajouter une section & 4 & Fiable, Facile & Pas de redondance. Ergonomique.\\
\hline
Modifier une section & 2 & Fiable, Facile & Coh{\'e}rence avec les
  enseignements. Ergonomique.\\
\hline
Supprimer une section & 4 & Fiable & Archivage. \\
\hline
Lister les sections & 4 & Rapide et Complet & Permettre de lister
  rapidement les mati{\`e}res existantes et assurer le
  listing de toutes les mati{\`e}res sans en oublier.\\
\hline
\end{tabular}
\begin{center}
{\'e}chelle de mesure de la priorit{\'e}:

\scalebox{0.5}{\includegraphics{images/echelle.jpg}}
\end{center}