\chapter{Introduction}

L'interface graphique de l'application web que nous allons d�velopp�e
s'appuie sur les derni�res technologies d'outils web.
Ce qui offre les avantages d'une norme internationale reconnue, d'une
compatibilit� avec tous les environnements Java, y compris les
navigateurs web.
L'utilisation de ces outils sont dues � la n�cessit� d'utiliser
une ref�rence commune de composants graphiques r�utilisables et ceci
dans un but technique et surtout pratique.
Ces technologies fournissent un code simple qui permet de cr�er une
interface tr�s pratique � l'aide des outils de cr�ation d'interface
graphique.

\section*{Objectifs }

\begin{itemize}
\item R�utisabilit�
\\
L'interface va former des briques pour son ind�pendance et sa
r�utilisabilit�.
On veut assurer la compl�te r�utilisabilit� de l'interface graphique
avec le moins de retouches de code source. 
\\
\item Simplicit�
\\
L'utilisation de notre interface par un autre d�veloppeur d'interface
ou un utilisateur qui construit lui-m�me une interface sera simple et
rapide.
Les balises vont communiquer entre elles par des messages bien
formalis�es.
Ces balises seront interchangeables. Si une partie de l'interface doit
changer pour cause d'une mise � jour o� d'ajout de fonctionnalit�, il
suffira de remplacer la balise obsol�te.
\\
\item Portabilit�
\\
L'interface sera �crit en JSP,HTML. Or ces technologies sont
maintenant tr�s largement portables.Ils sont utilisables aussi bien
sous Linux ou n'importe quel autre syst�me supportant java.
L'interface qui sera d�velopp�e aura le m�me aspect sur toutes les
platesformes.
On ne peut r�ver mieux en mati�re de portabilit�.
\\
\item Extensibilit�
\\
Les composants utilis�s par notre interface graphique seront
susceptibles d'�tre enrichis � volont�.
Donc il ne s'agit pas de r�ecrire une interface pour chaque nouveau
code.
\\
\item Normalisation:
\\
Notre interface est fond�e sur la norme ISO qu'est JSP et HTML et qui
sont des standards de l'industrie informatique.
Ceci garantit sa compatibilit� avec tous les environnements
d'exploitation et les navigateurs web comme Netscape
Navigator,Netscape Communicator, Internet Explorer ou Mozilla.
\end{itemize} 










